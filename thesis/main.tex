%TODO: Adjust 
\documentclass[11pt]{report}
%\renewcommand{\clearpage}{}

% Used indirectly by natlib
\usepackage{hyperref}
% Add url support for web pages etc.
\usepackage[square, numbers]{natbib}
% For custom spacing
\usepackage{setspace}
\setstretch{1.15}
% As required by the handbook.
\usepackage[a4paper, margin=25mm]{geometry}
% For the \say command
\usepackage{dirtytalk}
% Draft watermark.
\usepackage{draftwatermark}

\title{Kale - An attempt at an accessible and powerful visual programming
environment}
\author{Maciej Goszczycki - 16316981}
\date{}

\begin{document}

\maketitle
\tableofcontents
\clearpage

% Do not clear page after every chapter
\begingroup
\let\clearpage\relax

% *	The abstract must be an accurate reflection of what is in your report.
% * Your abstract must be self-contained, without abbreviations, footnotes, or
%   references. It should be a microcosm of the full report.
% * Your abstract must be 150-250 words written as one paragraph, and should
%   not contain displayed mathematical equations or tabular material.
% * Ensure that your abstract reads well and is grammatically correct.
% * The abstract must cover: motivation, the problem statement, the approach,
%   your results, and your conclusions.

\chapter{Abstract}
Contemporary text-based programming languages require the programer to be
constantly vigilant against syntax errors. This affects both novice and
experienced programmers. One obvious way of eliminating syntax errors is
get rid of text. While there exist many visual programming environments, many
eschew powerful editing capabilities for being beginner friendly.
This project explores how a Lisp-based visual programming environment might be
used to mitigate syntax issues while remaining attractive to both novices and
professional programmers.

\chapter{Introduction}

% Why would one care about the problem and the results? Cite appropriate
% references in this section. Explain the high-level, abstract problem that
% your project addresses. Explain what you are trying to achieve in a way that
% leads naturally into the next section.
\section{Motivation}
Syntax errors are a fact of life in the programming industry. Programmers from
novice\cite{Denny2011} to professional spend significant time fighting or
avoiding syntax errors. It is clear a visual programming environment such
%TODO: Cite scratch, cite syntax errors among professionals
Scratch could eliminate syntax errors. There exists a large body of existing
visual programming environments\cite{Beldie1983}, but most focus exclusively on
making programming accessible to children and young adults. Most professional
tools focus instead of error-detection, ignoring visual means of editing
programs as slow and cumbersome. It should be possible to create a visual
programming environment friendly to novices, but powerful enough to be taken
seriously by professionals. 


% Describe the technical problem needed to be solved in your project. Note that
% most projects solve both a more abstract, high-level problem and a specific,
% technical problem: your problem statement is the detailed technical problem
% (your motivation should cover the more abstract high-level problem). 
\section{Problem statement}
This project implements a web based visual programming environment, designed
from the ground up to fit many skill levels. Kale should demonstrate that a
drag and drop/blocks style interface can co-exist alongside a keyboard driven
professional focused editing experience. It should generate usable code,
runnable from within the interface.

\section{Approach}

%TODO: Write these out
\begin{itemize}
	\item Implementing the rendering engine
	\item Keyboard shortcuts
	\item Drag and drop functionality
	\item Working interpreter
\end{itemize}
Lisp's highly uniform syntax, makes it the perfect 

\section{Metrics}
\section{Project}

As part of the development process I contributed patches to two different
open-source projects, including adding \texttt{\#rrggbbaa} notation support
to the Popmotion library\cite{github-pr-popmotion} and updating
\say{styled-components} TypeScript typings to the a new major 5.0
version\cite{github-pr-styled-components}.

\chapter{Technical Background}
\section{Topic material}
\section{Technical material}

\chapter{The Problem}

•	Implement KaleLisp alongside Kale, a simple Kale-specific Lisp-like
language.
•	Have Kale run in the browser (but other platforms would be great)
•	It should have enough functionality to write and edit basic KaleLisp programs
easily and efficiently.
•	Kale should be easily discoverable and learnable at different skill levels.
•	A proficient user should be able to create a KaleLisp expression without
touching the mouse.
•	Kale should work equally well in colour as greyscale.


The programming language which Kale users will be writing should map well to
the UI, but be powerful enough to be able to express complex programs. It need
not necessarily be an existing language, it fact it might not be desirable to
make it one.

A proficient User should be able to write programs and navigate Kale without
having to reach for the mouse. This might involve mapping Kale concepts to
standard shortcut keys as well as developing new ones. It should be possible
for users of existing programming environments to adjust to Kale with relative
ease.

Novice users should be able to to construct Kale programs in a simple and
initiative manner, on every device form-factor, whether touch-screen or
mouse-driven. It should be simple for them to discover new functionality and
edit existing programs without knowledge of higher level operations.

\chapter{The Solution}

\section{Novice Users}



\section{Discoverability}

\section{Implementation}

\chapter{Evaluation}
\chapter{Conclusion}

\endgroup % Ends the clearpage re-definition group
\renewcommand*{\bibfont}{\raggedright} % Make the reference ragged right.
\bibliographystyle{unsrtnat} % List in order of mention.
\bibliography{bibliography.bib}


\end{document}
