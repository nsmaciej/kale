\chapter{The Solution}

\section{Expression structure}

\section{Drag and drop}

One of the main changes Kale introduces compared to a normal Lisp editor is
drag and drop. It lets novice users effectively manipulate Kale programs,
without prior knowledge of commands like copy and paste. Kale distinguished
between two types of drag and drop: Replacement and Insertion. 

\section{Layout}

One of the key ideas in Kale is that of structural underlines. As a
fundamental unit of nesting, parentheses present a number of challenges to a
user using drag and drop.
%TODO Talk about in inline, non-inline, underlines optional bubbles.

\section{Command structure}

Because Kale does not operate on elementary text elements, even the most
basic
commands can be much higher-level than ordinary editors, since instead of a
current cursor position, Kale operates on the selected expression. Even so,
figuring out the correct set of commands proved to be a challenge.

A good example of the difficulties that arise are the arrow key commands
\ak{^} \ak{v} \ak{<} \ak{>}. Initially these were implemented as
fundamental tree operations: \say{Select parent}, \say{Select first child},
\say{Select left sibling}, and \say{Select right sibling} respectively.
While logical, these operations were unintuitive to every user
Kale was shown too, including the author. In the end \say{smart selection} was
implemented, where \ak{<} and \ak{>} use the pre-order%
%
\footnote{The preorder traversal algorithm traverses the parent node first,
then traverses the left and right tree by calling the itself on each.}
%
traversal, while \ak{^} and \ak{v}
pre-order traverse only non-inline non-list expressions, mirroring the visual
line motion a normal cursor might make.

\section{Discoverability}
\setlength\intextsep{0pt}
\begin{wrapfigure}[21]{R}{5.5cm}
\includegraphics[width=5.5cm]{figures/menu.png}
\caption{Kale's Context Menu}
\end{wrapfigure}
Most commands can be accessed in at least three ways, through their dedicated
keyboard shortcut, the context menu and the top-level editor menu. In case the
user is keen on using the keyboard shortcuts but forgot a specific command,
each editor menu-item shows the corresponding keyboard shortcut, and each
shortcut can be triggered whilst the context menu is open. These keyboard
shortcut indicators are also placed throughout Kale to help with discovering
shortcuts for the Clipboard List or the Function Search menu.


\section{High-level manipulation}

\subsection{List merging}

\subsection{Clipboard list}
\begin{wrapfigure}[11]{R}{0.pt}
	\includegraphics[width=5.5cm]{figures/clipboard}
	\caption{The Clipboard List}
\end{wrapfigure}

Relinquishing control over the elementary elements of a program might
potentially make more high-level manipulation problematic. To remedy this
Kale needed to provide a better way of transforming expressions. Andrew Blinn's
Fructure \cite{Fructure} environment tackles this by letting user colour
expressions, then
using a special transformation mode where the user can enter new expression or
use one of the previously coloured ones. While imbued with a certain sense of
mathematical purity, this approach deals poorly with more complex refactorings
and requires colour vision.

Kale's solution to this comes in the form of the \textbf{Clipboard List}. The
Clipboard List is a stack of expressions, shown on the right-hand side of the
screen. Kale provides a \hyperref[cmd:copy]{\say{Copy}} \keys{C} command, which
copies the currently
selected expression to the top of the stack. To facilitate more destructive
refactoring, Kale provides a palette of deletion commands:

\begin{itemize}[noitemsep]
	\item \hyperref[cmd:delete]{\say{Delete}} \keys{\backspace}
	\item \hyperref[cmd:cut]{\say{Cut}} \keys{X}
	\item \hyperref[cmd:delete_blank]{\say{Delete and Add Space}} \keys{R}
	\item \hyperref[cmd:cut_blank]{\say{Cut and Add Space}} \keys{S}
\end{itemize}

\subsection{Smart space}
%TODO: Show a sequence explaining how this works.
%TODO: Mention how this isn't a problem in scratch since you can't add new
% arguments.
In contemporary programming environments various punctuation marks are used to
create new expressions. This presents a challenge for an editor like Kale, how
to create new expressions. At first this was implemented as a set of
expression kind specific operations like \say{Create new child}, \say{Create
new sibling}, and \say{Create new line}. However this proved unintuitive as
unlike classical punctuation, the keys for these operations did not correspond
to any character on the screen, thus making it hard to memorise the shortcuts.

The solution Kale is using is named \textbf{Smart Space}. Smart space is a
high-level operation that attempts to perform a reasonable action no matter the
selection.
\begin{itemize}[noitemsep]
	\item If a function is selected, create a new child space in the first
argument.
	\item If a space is selected, use the \hyperref[cmd:up_down]{\say{Move Up}}
	\keys{\shift + P} operation.
	\item Otherwise create a new sibling space to the right of the selection, for
example creating a new argument.
\end{itemize}

Note that this does not cover \say{Create new line} operation, so this option
is still exists as
\hyperref[cmd:new_line]{\say{New Line Below}} \keys{N} /
\hyperref[cmd:new_line]{\say{New Line Below}} \keys{\shift + N}.

\begin{figure}
	\begin{minipage}{0.5\linewidth}
	\centering
	\includegraphics[width=\linewidth]{figures/clipboard.png}
	\caption{TODO: Before smart space}
	\end{minipage}
	\qquad
	\begin{minipage}{0.5\linewidth}
	\centering
	\includegraphics[width=\linewidth]{figures/clipboard.png}
	\caption{TODO: After smart space}
	\end{minipage}
\end{figure}

\section{Novice Users}
A very powerful operation for manipulating programs is drag and drop. It allows
for direct manipulation of Kale expressions, without requiring any knowledge of
the keybindings.

\begin{itemize}
	\item Mention touch-screen support.
	\item The toy box / Blocks area.
	\item Drag and drop details
\end{itemize}

\section{Professional users}

TODO: Move this around, mention command structure again,
%TODO: Talk about the different expr types

High level operations and complex programs necessitate an intelligent layout
engine which is able to cleanly layout arbitrary code.
%TODO: Write a quick slurp program

The conditions for an inline expressions are as follows:

\begin{itemize}[noitemsep]
	\item Expression is a literal\footnote{In the future this might not be the
only condition as more literal types get added}
	\item Expression is a variable name
	\item Expression is a call with no arguments
	\item Expression is a call and the following are true
	\begin{enumerate}[noitemsep]
		\item Every argument is also inline
		\item The sum total length of the arguments is below 300 pixels
		\item The hight of the expression tree of every argument is below 4
	\end{enumerate}
\end{itemize}

\section{Font choice}

\begin{itemize}
	\item Open source proportional font (cite)
	\item Something normally difficult to achieve in normal text-based grid-based
	editors.
	
\end{itemize}