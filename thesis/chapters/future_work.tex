\section{Future Work}

\subsection{User studies}
The main concern of this project was creating a property of the Kale
environment, in the future, it would be beneficial to perform user studies on
the Kale and the various design choices made along the way. Somethings that
would be investigated might be

\begin{itemize}[noitemsep]
	\item The value of the structured underlines.
	%TODO: Cite that study
	\item The effectiveness of Kale, in transitioning children away from scratch.
	\item User experience of professional programmers.
\end{itemize}

\subsection{Drag-and-drop improvements}
Currently due to the limitations of drag-and-drop implementation, it is
impossible to insert new expression as the last sibling of expression lists
and function calls.

\subsection{Integrated debugging}
%TODO: Include a screenshot of a highlighted expr to show how breakpoints might
% work
Right now the runtime capabilities provided by Kale are quite basic. While
effort was put into making the error messages somewhat descriptive, there are 
are advanced debugging capabilities available. This is unfortunate as
expression-oriented selection lends itself nicely to visualising the currently
execution expression or setting breakpoints. Modern text-oriented debuggers use
inline markers to try and indicate which expression the debugger is on, a
feature called \fnurl{\say{inline breakpoints}}
{https://developers.google.com/web/updates/2019/04/devtools\#inline}.

\subsection{Fluid field editor}
\say{Fluid entry} is a concept for simplifying entering new expressions. Right
now to turn a space into another expression the user needs to either enter the
\hyperref[soln:space_popover]{\say{space popover}} or memorise a set of
shortcuts like
\hyperref[cmd:make_var]{\say{Make a Variable...}}~\keys{V} or
\hyperref[cmd:make_var]{\say{Turn into a Function Call...}}~\keys{F}.
These shortcuts are currently essential to efficiently creating new Kale
expressions, but for many expressions they feel like an extra step that would
not be necessary under a normal text editor.

The idea behind the \say{fluid field editor} would be to eliminate these
shortcuts, instead delaying the decision on what type of expression
should be created until the user has typed a character.
\medskip
\begingroup
\def\arraystretch{1.2}
\begin{center}
\begin{tabular}{ l l }
	\textbf{Typed character} & \textbf{Resulting expression} \\
	\hline
	\keys{"} or \keys{'} & Text literal \\
	\keys{0} -- \keys{9} & Number literal \\
	\keys{\shift + A} -- \keys{\shift + Z} & Function call \\
	\keys{A} -- \keys{Z} & Variable expression \\
\end{tabular}
\end{center}
\endgroup
\medskip

\subsection{Removing modality}
Removing the \hyperref[soln:field_editing]{field editing} feature might be
worth investigation. This could be achieved by implicitly entering the field
editing mode when changing the selection. \ak{<}~and~\ak{>} could then always
move the cursor inside the current field, similarly to standard text editors,
once. If the cursor reaches the end of a field, the next expression would be
selected, following the current default behaviour.